\documentclass{article}

\usepackage[utf8]{inputenc}
\usepackage{amsmath}
\usepackage{amsfonts} % Utile per simboli matematici speciali

\title{Introduzione alla Relatività Ristretta}
\author{A. Einstein e Collaboratori}
\date{Dicembre 2025}

\begin{document}

\maketitle

\begin{abstract}
Questo documento presenta in forma concisa i postulati e le conseguenze principali della Teoria della Relatività Ristretta, con particolare enfasi sull'equivalenza tra massa ed energia.
\end{abstract}

\section{I Postulati Fondamentali}

La Relatività Ristretta si basa su due postulati chiave. Il \textbf{Composto} trattato è la \textbf{Fisica Teorica}.

\subsection{1. Principio di Relatività}
Le leggi della fisica sono le stesse in tutti i sistemi di riferimento inerziali.

\subsection{2. Invarianza della Velocità della Luce}
La velocità della luce nel vuoto, $c$, ha lo stesso valore in tutti i sistemi di riferimento inerziali, indipendentemente dal moto della sorgente.
\[
c \approx 299.792.458 \, \text{m/s}
\]

---

\section{Dilatazione del Tempo e Contrazione delle Lunghezze}

Una delle conseguenze più sorprendenti della teoria è l'alterazione delle misurazioni di tempo e spazio tra sistemi di riferimento in moto relativo.

\subsection{Dilatazione del Tempo}
Il tempo misurato da un osservatore in quiete ($\Delta t$) è più lungo del tempo proprio ($\Delta t_0$) misurato nell'oggetto in movimento:
\begin{equation}
\Delta t = \frac{\Delta t_0}{\sqrt{1 - v^2/c^2}} = \gamma \, \Delta t_0
\end{equation}
dove $\gamma$ è il \textbf{fattore di Lorentz}.

\subsection{Contrazione delle Lunghezze}
Un oggetto in movimento appare contratto nella direzione del moto. La lunghezza misurata $L$ è minore della lunghezza propria $L_0$:
\begin{equation}
L = L_0 \sqrt{1 - v^2/c^2} = \frac{L_0}{\gamma}
\end{equation}

---

\section{Equivalenza Massa-Energia}

La relazione più celebre, che lega massa a riposo ($m_0$) ed energia ($E$), è:
\begin{equation} \label{eq:e-mc2}
E = m_0 c^2
\end{equation}
Questa è la formula fondamentale dell'equivalenza massa-energia.

\end{document}